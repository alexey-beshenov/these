\documentclass{article}

\usepackage{fullpage}

% % % % % % % % % % % % % % % % % % % % % % % % % % % % % %
% CLEAN UP THIS MESS AT SOME POINT :-(

\usepackage[leqno]{amsmath}
\usepackage{amssymb}
\usepackage{graphicx}

\usepackage{scrextend}
\usepackage{epigraph}

\renewcommand{\thefootnote}{\ifcase\value{footnote}\or*\or**\or***\or****\or($\infty$)\fi}
\usepackage{perpage}
\MakePerPage{footnote}

\usepackage{array}
\newcolumntype{x}[1]{>{\centering\hspace{0pt}}p{#1}}

\DeclareMathOperator*{\colim}{co{\lim}}
\newcommand{\dirlim}{\varinjlim}
\newcommand{\invlim}{\varprojlim}

\newcommand{\tikzpb}{\ar[phantom,pos=0.2]{dr}{\text{\large$\lrcorner$}}}
\newcommand{\tikzpbur}{\ar[phantom,pos=0.2]{dl}{\text{\large$\llcorner$}}}

\newcommand{\term}{\textbf}

\usepackage[utf8]{inputenc}

\usepackage{stmaryrd}

\usepackage{cancel}

\newcommand{\DB}{{\mathcal{D}\text{-\foreignlanguage{russian}{Б}}}}
\newcommand{\open}{\underset{\mathrm{open}}{\hookrightarrow}}
\newcommand{\closed}{\underset{\mathrm{closed}}{\hookrightarrow}}

\newcommand{\tcol}[2]{{#1 \choose #2}}

\newcommand{\homot}{\simeq}
\newcommand{\isom}{\cong}
% \newcommand{\cH}{\mathcal{H}}
\renewcommand{\hom}{\mathrm{hom}}
\renewcommand{\div}{\mathop{\mathrm{div}}}
\renewcommand{\Im}{\mathop{\mathrm{Im}}}
\renewcommand{\Re}{\mathop{\mathrm{Re}}}
\newcommand{\id}[1]{\mathrm{id}_{#1}}
\newcommand{\idid}{\mathrm{id}}

\newcommand{\ZG}{{\ZZ G}}
\newcommand{\ZH}{{\ZZ H}}

% \newcommand{\quiso}{\stackrel{\mathrm{q.iso}}{\cong}}
\newcommand{\quiso}{\simeq}

\newcommand{\personality}[1]{{\sc #1}}

\newcommand{\mono}{\rightarrowtail}
\newcommand{\epi}{\twoheadrightarrow}

\DeclareMathOperator{\Ann}{Ann}
\DeclareMathOperator{\CH}{CH}
\DeclareMathOperator{\Cl}{Cl}
\DeclareMathOperator{\Cone}{Cone}
\DeclareMathOperator{\Cop}{Cop}
\DeclareMathOperator{\Der}{Der}
\DeclareMathOperator{\Div}{Div}
\DeclareMathOperator{\D}{D}
\DeclareMathOperator{\End}{End}
\DeclareMathOperator{\Eq}{Eq}
\DeclareMathOperator{\Ext}{Ext}
\DeclareMathOperator{\Frac}{Frac}
\DeclareMathOperator{\Frob}{Frob}
\DeclareMathOperator{\Funct}{Funct}
\DeclareMathOperator{\Fun}{Fun}
\DeclareMathOperator{\GL}{GL}
\DeclareMathOperator{\Gal}{Gal}
\DeclareMathOperator{\Gr}{Gr}
\DeclareMathOperator{\Hol}{Hol}
\DeclareMathOperator{\Hom}{Hom}
\DeclareMathOperator{\Id}{Id}
\DeclareMathOperator{\Inn}{Inn}
\DeclareMathOperator{\Isom}{Isom}
\DeclareMathOperator{\Lie}{Lie}
\DeclareMathOperator{\Map}{Map}
\DeclareMathOperator{\Mat}{Mat}
\DeclareMathOperator{\Mor}{Mor}
\DeclareMathOperator{\Nat}{Nat}
\DeclareMathOperator{\Ob}{Ob}
\DeclareMathOperator{\Pic}{Pic}
\DeclareMathOperator{\Sing}{Sing}
\DeclareMathOperator{\Spec}{Spec}
\DeclareMathOperator{\Sp}{Sp}
\DeclareMathOperator{\Stab}{Stab}
\DeclareMathOperator{\Tot}{Tot}
\DeclareMathOperator{\YExt}{YExt}
\DeclareMathOperator{\ad}{ad}
\DeclareMathOperator{\card}{card}
\DeclareMathOperator{\codim}{codim}
\DeclareMathOperator{\coeq}{coeq}
\DeclareMathOperator{\coim}{coim}
\DeclareMathOperator{\coker}{coker}
\DeclareMathOperator{\cor}{cor}
\DeclareMathOperator{\eq}{eq}
\DeclareMathOperator{\ev}{ev}
\DeclareMathOperator{\fchar}{char}
\DeclareMathOperator{\gr}{gr}
\DeclareMathOperator{\im}{im}
\DeclareMathOperator{\mcd}{mcd}
\DeclareMathOperator{\ord}{ord}
\DeclareMathOperator{\pr}{pr}
\DeclareMathOperator{\rel}{rel}
\DeclareMathOperator{\res}{res}
\DeclareMathOperator{\rk}{rk}
\DeclareMathOperator{\sgn}{sgn}
\DeclareMathOperator{\supp}{supp}
\DeclareMathOperator{\trdeg}{trdeg}
\DeclareMathOperator{\tr}{tr}
\newcommand{\RHom}{R\!\Hom}
\newcommand{\RiHom}{R\underline{\Hom}}
\newcommand{\iHom}{\underline{\Hom}}

\newcommand{\legendre}[2]{\left(\frac{#1}{#2}\right)}

\newcommand{\dfn}{\mathrel{\mathop:}=}
\newcommand{\rdfn}{=\mathrel{\mathop:}}

\usepackage[rgb]{xcolor}
% \definecolor{myurlcolor}{rgb}{1.0,0.0,0.4}
% \definecolor{mylinkcolor}{rgb}{0.0,0.4,1.0}
% \definecolor{mycitecolor}{rgb}{0.0,0.4,1.0}
\definecolor{myurlcolor}{rgb}{0.0,0.0,0.0}
\definecolor{mylinkcolor}{rgb}{0.0,0.0,0.0}
\definecolor{mycitecolor}{rgb}{0.0,0.0,0.0}
\definecolor{shadecolor}{rgb}{0.89,0.88,0.80}
\definecolor{gray}{rgb}{0.7,0.7,0.7}
\definecolor{lightgray}{rgb}{0.8,0.8,0.8}

\newcommand{\refref}[2]{\hyperref[#2]{#1~\ref*{#2}}}
\newcommand{\eqnref}[1]{\hyperref[#1]{(\ref*{#1})}}

\newcommand{\tos}{\!\!\to\!\!}

\makeatletter
\newcommand{\dashfill}{%
\leavevmode \cleaders \hb@xt@ .50em{\hss -\hss }\hfill \kern \z@
% original \dotfill macro:
% \leavevmode \cleaders \hb@xt@ .44em{\hss .\hss }\hfill \kern \z@
}
\makeatother

\newcommand{\cequiv}{\simeq}

\usepackage[sc]{mathpazo}
\linespread{1.05}         % Palladio needs more leading (space between lines)
\usepackage[T2A,T1]{fontenc}

\usepackage{helvet}
\usepackage{titlesec}
\titleformat{name=\chapter}[display]
  {\normalfont\sffamily\huge\bfseries}
  {\chaptertitlename\ \thechapter}{20pt}{\Huge}
\titleformat{\section}
  {\normalfont\sffamily\Large\bfseries}
  {\thesection}{1em}{}
\titleformat{\subsection}
  {\normalfont\sffamily\large\bfseries}
  {\thesection}{1em}{}

\newcommand{\CC}{\mathbb{C}}
\newcommand{\FF}{\mathbb{F}}
\newcommand{\HH}{\mathbb{H}}
\newcommand{\NN}{\mathbb{N}}
\newcommand{\PP}{\mathbb{P}}
\newcommand{\QQ}{\mathbb{Q}}
\newcommand{\RR}{\mathbb{R}}
\newcommand{\ZZ}{\mathbb{Z}}
\renewcommand{\AA}{\mathbb{A}}

\makeatletter
\newcommand\xleftrightarrow[2][]{%
  \ext@arrow 9999{\longleftrightarrowfill@}{#1}{#2}}
\newcommand\longleftrightarrowfill@{%
  \arrowfill@\leftarrow\relbar\rightarrow}
\makeatother

% % % % % % % % % % % % % % % % % % % % % % % % % % % % % %

\usepackage{amsthm}

\newcommand{\examplesymbol}{$\blacktriangle$}
\renewcommand{\qedsymbol}{$\blacksquare$}

\newtheoremstyle{myplain}
  {\topsep}   % ABOVESPACE
  {\topsep}   % BELOWSPACE
  {\itshape}  % BODYFONT
  {0pt}       % INDENT (empty value is the same as 0pt)
  {\bfseries} % HEADFONT
  {.}         % HEADPUNCT
  {5pt plus 1pt minus 1pt} % HEADSPACE
  {\thmnumber{#2}. \thmname{#1}\thmnote{ (#3)}}   % CUSTOM-HEAD-SPEC 

\newtheoremstyle{myplainnameless}
  {\topsep}   % ABOVESPACE
  {\topsep}   % BELOWSPACE
  {\normalfont}  % BODYFONT
  {0pt}       % INDENT (empty value is the same as 0pt)
  {\bfseries} % HEADFONT
  {.}         % HEADPUNCT
  {5pt plus 1pt minus 1pt} % HEADSPACE
  {\thmnumber{#2}}   % CUSTOM-HEAD-SPEC 

\newtheoremstyle{mydefinition}
  {\topsep}   % ABOVESPACE
  {\topsep}   % BELOWSPACE
  {\normalfont}  % BODYFONT
  {0pt}       % INDENT (empty value is the same as 0pt)
  {\bfseries} % HEADFONT
  {.}         % HEADPUNCT
  {5pt plus 1pt minus 1pt} % HEADSPACE
  {\thmnumber{#2}. \thmname{#1}\thmnote{ (#3)}}   % CUSTOM-HEAD-SPEC

%%%%%%%%%%%%%%%%%%%%%%%%%%%%%%%%%%%%%%%%%%%%%%%%%%%%%%%%%%%%%%%%%%%%%%

\theoremstyle{myplain}
\newtheorem{proposition}{Proposition}[section]
\newtheorem{fact}{Fact}[section]
\newtheorem{lemma}[proposition]{Lemma}
\newtheorem{observation}[proposition]{Observation}
\newtheorem{question}[proposition]{Question}

\newtheorem{theorem}[proposition]{Theorem}
\newtheorem{conjecture}[proposition]{Conjecture}
\newtheorem{corollary}[proposition]{Corollary}

\theoremstyle{myplainnameless}
\newtheorem{nameless}[proposition]{}

\theoremstyle{mydefinition}
\newtheorem{examplex}[proposition]{Example}
\newenvironment{example}
  {\pushQED{\qed}\renewcommand{\qedsymbol}{\examplesymbol}\examplex}
  {\popQED\endexamplex}

\newtheorem*{examplexx}{Example}
\newenvironment{example*}
  {\pushQED{\qed}\renewcommand{\qedsymbol}{\examplesymbol}\examplexx}
  {\popQED\endexamplexx}

\newtheorem{definition}[proposition]{Definition}
\newtheorem{remark}[proposition]{Remark}

\makeatletter
\newcommand{\xRightarrow}[2][]{\ext@arrow 0359\Rightarrowfill@{#1}{#2}}
\makeatother

%%%%%%%%%%%%%%%%%%%%%%%%%%%%%%%%%%%%%%%%%%%%%%%%%%%%%%%%%%%%%%%%%%%%%%

\newcommand{\Et}{\mathop{\text{\rm \'Et}}}

\DeclareFontFamily{OT1}{pzc}{}
\DeclareFontShape{OT1}{pzc}{m}{it}{<-> s * [1.1] pzcmi7t}{}
\DeclareMathAlphabet{\mathpzc}{OT1}{pzc}{m}{it}

\newcommand{\separator}{\vspace{1em}\hrule\vspace{1em}}

\usepackage{pifont}
\newcommand{\categ}[1]{\text{\bf #1}}
\newcommand{\vcateg}{\mathcal}
\newcommand{\bone}{{\boldsymbol 1}}
\newcommand{\bDelta}{{\boldsymbol\Delta}}
\newcommand{\bsquare}{\textrm{\ding{114}}}
\newcommand{\bR}{{\mathbf{R}}}

% \usepackage{ulem}
\newcommand{\uuline}[1]{\underline{\underline{#1}}}

\renewcommand{\mathcal}{\mathpzc}

\makeatletter
\def\iddots{\mathinner{\mkern1mu\raise\p@
\vbox{\kern7\p@\hbox{.}}\mkern2mu
\raise4\p@\hbox{.}\mkern2mu\raise7\p@\hbox{.}\mkern1mu}}
\makeatother


\newcommand{\ssincl}{\reflectbox{\rotatebox[origin=c]{45}{$\subseteq$}}}

\newcommand{\vsupseteq}{\reflectbox{\rotatebox[origin=c]{-90}{$\supseteq$}}}
\newcommand{\vin}{\reflectbox{\rotatebox[origin=c]{90}{$\in$}}}

\newcommand{\Ga}{\mathbb{G}_\mathrm{a}}
\newcommand{\Gm}{\mathbb{G}_\mathrm{m}}

% \usepackage[a-1b]{pdfx}
\usepackage{hyperref}
\hypersetup{colorlinks=true,linkcolor=mylinkcolor,urlcolor=myurlcolor,citecolor=mycitecolor,pdfencoding=unicode}

\renewcommand{\O}{\mathcal{O}}

\usepackage[russian,spanish,francais,dutch,english]{babel}

\usepackage{tikz-cd}
\usetikzlibrary{babel}
\usetikzlibrary{arrows}
\usetikzlibrary{calc}


% Not numbered
\theoremstyle{plain}
\newtheorem*{drawback*}{Drawback}
\newtheorem*{theorem*}{Theorem}
\newtheorem*{conjecture*}{Conjecture}
\newtheorem*{proposition*}{Proposition}
\newtheorem*{question*}{Question}
\newtheorem*{mygoal*}{My goal}

\usepackage[numbers]{natbib}

\date{Leiden Mathematical Institute, December 4, 2017}
\author{Alexey Beshenov (cadadr@gmail.com)}
\title{Zeta-values of arithmetic schemes at negative integers and Weil-étale cohomology}

\begin{document}

{\normalfont\sffamily\bfseries \maketitle}

\setcounter{section}{-1}
\section{Motivation}

Let us consider an \term{arithmetic scheme} $X$:

\[ \begin{tikzcd}
    X \ar{d}{\begin{array}{l}\text{separated}\\\text{of finite type}\end{array}} \\
    \Spec \ZZ
\end{tikzcd} \]

We may associate to it the corresponding \term{zeta function}, which is defined
by the Euler product
$$\zeta (X, s) \dfn \prod_{x\in X_0} \frac{1}{1 - N (x)^{-s}},$$
where $X_0$ denotes the set of closed points of $X$, and $N (x)$ is the
cardinality of the residue field of $x$. For instance, if $X = \Spec \ZZ$,
we recover the usual Riemann zeta function $\zeta (s)$; more generally, if
$X = \Spec \O_F$ is the spectrum of a number ring, we obtain the Dedekind zeta
function $\zeta_F (s)$. The above product converges for $\Re s > \dim X$, and
from now on I will also make the following assumption.

\begin{conjecture*}
  $\zeta (X,s)$ has a meromorphic continuation to the whole complex plane, which
  we also denote by $\zeta (X,s)$.
\end{conjecture*}

For each integer $n\in \ZZ$, one might ask about the following two quantities:

\begin{enumerate}
\item[1)]
  $d_n \dfn \ord_{s = n} \zeta (X,s) \dfn \text{the vanishing order at }s=n$;

\item[2)] the corresponding \term{special value}, i.e. the leading term of the
  Taylor expansion at $s = n$:
  $$\zeta^* (X,n) \dfn \lim_{s\to n} (s-n)^{-d_n} \, \zeta (X,s).$$
\end{enumerate}

Cohomological interpretation of special values may be traced to 1839 when
Dicichlet published the \term{class number formula}. The Dedekind zeta function
$\zeta_F (s)$ has a zero of order $d_0 = r_1 + r_2 - 1$ at $s = 0$, where $r_1$
and $2\,r_2$ is the number of real and complex places of $F$, and the
corresponding special value is
$$\zeta_F^* (0) = -\frac{h_F}{\#\mu_F}\,R_F,$$
where $h_F$ is the class number, $\mu_F = (\O_F^\times)_\text{\it tors}$ is the
group of roots of unity in $F$, and $R_F$ is the regulator. The above formula
may be written as
\[
  \zeta^* (\Spec \O_F, 0) =
  -\frac{\# H^1 (\Spec \O_F, \mathbb{G}_m)}{\# H^0 (\Spec \O_F, \mathbb{G}_m)_\text{\it tors}}\,R_F =
  -\frac{\# H^0 (\Spec \O_F, \ZZ^c (0))}{\# H^{-1} (\Spec \O_F, \ZZ^c (0))_\text{\it tors}}\,R_F,
\]
where $\ZZ^c (0) \isom \mathbb{G}_m [1]$ (see below).

\pagebreak

In this sense, the first results about finite generation of certain motivic
cohomology groups were obtained in the XIX century:

\begin{itemize}
\item finiteness of the class group is finiteness of
  $\# H^0 (\Spec \O_F, \ZZ^c (0))$;

\item Dirichlet's unit theorem says that $H^{-1} (\Spec \O_F, \ZZ^c (0))$ is a
  group of finite rank $r_1 + r_2 - 1$.
\end{itemize}

Here I won't get into the details about the classical conjectures of
Lichtenbaum, Be\u{\i}linson, and others; instead I refer to the survey
\cite{Kahn-2005}. More recently, Lichtenbaum envisioned the existence of certain
cohomology theory, named \term{Weil-étale cohomology}, that (conjecturally)
encodes the information about vanishing orders and special values of
$\zeta (X,s)$. Here is a very brief history of the subject.

\begin{itemize}
\item Lichtenbaum first studied Weil-étale cohomology for varieties over finite
  fields in \cite{Lichtenbaum-05}. Further results were obtained by Thomas
  Geisser in \cite{Geisser-04-finite}.

\item In \cite{Lichtenbaum-09-number-rings} Lichtenbaum considered the case of
  number rings $X = \Spec \O_F$ and $n = 0$.

\item Baptiste Morin constructed in \cite{Morin-14} Weil-étale cohomology for
  the case of regular, proper $X$ and $n = 0$.

\item Matthias Flach and Baptiste Morin generalized this in
  \cite{Flach-Morin-16} to all $n \in \ZZ$, again for regular, proper $X$.
\end{itemize}

I am investigating the following situation.

\begin{mygoal*}
  Construct and study Weil-étale cohomology for an arbitrary arithmetic scheme
  $X$ and $n < 0$.
\end{mygoal*}

So from now on, $X$ will denote any arithmetic scheme and $n$ will denote a
\emph{strictly negative} integer. Removing the assumptions on $X$ in theory
should make everything harder, but at the same time, restricting the attention
to the case of $n < 0$ simplifies many things. I am following the ideas of Flach
and Morin, and in particular, when $X$ is regular and proper, the constructions
and conjectures that I am going to describe coincide with theirs. Not out of
immodesty, but due to the lack of time, I will focus on my case and outline the
involved tools and definitions.

% % % % % % % % % % % % % % % % % % % % % % % % % % % % % %

\section{Motivic cohomology}

There are several constructions of motivic cohomology. The one that is suitable
for arithmetic schemes originates from the seminal paper of Spencer Bloch on
higher Chow groups \cite{Bloch-1986}. Bloch's ideas have been further developed
by Marc Levine and other mathematicians, and the corresponding techniques have
been also applied to arithmetic schemes (see Geisser's survey
\cite{Geisser-survey}).

Geisser in \cite{Geisser-10} introduced \term{dualizing cycle complexes}, which
is a certain variation of Bloch's cycle complexes:
$$(X,n) \rightsquigarrow \ZZ^c (n) \text{, a complex of sheaves on }X_\text{\it ét}.$$
For those familiar with motivic cohomology, if $X$ is an equidimensional scheme
of dimension $d$, then
$$\ZZ^c (n) = \ZZ (d-n) [2d],$$
where the right hand side is the sheaf defined from Bloch's cycle complex.
The same relation would hold with the motivic complex of Voevodsky, if we work
with smooth schemes over a field. This is not the case of our interest, however.

Another important thing to keep in mind is that the (hyper)cohomology of
$\ZZ^c (n)$ behaves very much like \term{Borel--Moore homology} in the
topological setting. Namely, if $X$ is a locally compact topological space,
$U \subset X$ is its open subspace, and $Z \dfn X\setminus U$ is the
corresponding closed complement, in such a case I will say that we have an
\term{open-closed decomposition}
$$U \hookrightarrow X \leftarrow Z$$
This gives a distinguished triangle in the derived category of abelian groups
$$R\Gamma_{BM} (Z, \ZZ) \to R\Gamma_{BM} (X, \ZZ) \to R\Gamma_{BM} (U, \ZZ) \to \cdots [1]$$
simply because by definition (well, not the original one of Borel and Moore, but
the one of Verdier),
$$R\Gamma_{BM} (X,\ZZ) \dfn R\Gamma (X, p^! \underline{\ZZ}) \isom \RHom (R\Gamma_c (X,\ZZ), \ZZ),$$
where $p\colon X\to \ast$ is the projection to a point. So the above
distinguished triangle is nothing more than the Verdier dual of the
distinguished triangle for cohomology with compact support
$$R\Gamma_c (U,\ZZ) \to R\Gamma_c (X,\ZZ) \to R\Gamma_c (Z,\ZZ) \to \cdots [1]$$

Similarly, for Geisser's complexes $\ZZ^c (n)$, an open-closed decomposition of
schemes gives a distinguished triangle
\[
  R\Gamma (Z_\text{\it ét}, \ZZ^c (n)) \to
  R\Gamma (X_\text{\it ét}, \ZZ^c (n)) \to
  R\Gamma (U_\text{\it ét}, \ZZ^c (n)) \to \cdots [1]
\]
The intuition behind this is that initially, $\ZZ^c (n)$ has an ad hoc
definition as a cycle complex, but then it is possible to identify it as
a dualizing complex in certain arithmetic contexts. This is what Geisser does in
\cite{Geisser-10}.

In general, not much is known about the cohomology of $\ZZ^c (n)$, and to
proceed, we need to make the following assumption.

\vspace{1em}

\noindent\textbf{Conjecture $\mathbf{L}^c (X_\text{\it ét}, n).$}
\emph{The (hyper)cohomology groups $H^i (X_\text{\it ét}, \ZZ^c (n))$ are
  finitely generated.}

% % % % % % % % % % % % % % % % % % % % % % % % % % % % % %

\section{Weil-étale complexes}

I don't have enough time to enter in all the gory details of the construction of
Weil-étale complexes (for this, I refer to the appendix), so let me state what
kind of object it is.

\vspace{1em}

\noindent\textbf{Output of the construction}.
{\it Assume that the conjecture $\textbf{L}^c (X_\text{\it ét}, n)$ holds.

\begin{enumerate}
\item[1)] There exists a perfect object in the derived category of abelian
  groups
  $$R\Gamma_\text{\it W,c} (X, \ZZ (n)),$$
  which we call \term{Weil-étale cohomology with compact support}. That is, the
  corresponding cohomology groups
  $$H^i_\text{\it W,c} (X, \ZZ (n)) \dfn H^i (R\Gamma_\text{\it W,c} (X, \ZZ (n)))$$
  are finitely generated and vanish for almost all $i$.

\item[2)] After tensoring with $\RR$, this complex splits (non-canonically) as
  \[
    R\Gamma_\text{\it W,c} (X,\ZZ(n)) \otimes \RR \isom
    \RHom (R\Gamma (X_\text{\it ét}, \ZZ^c (n)), \RR) [-1]
    \oplus
    R\Gamma_c (G_\RR, X (\CC), (2\pi i)^n\,\RR) [-1].
  \]
\end{enumerate}}

\vspace{1em}

I already briefly explained what is $R\Gamma (X_\text{\it ét}, \ZZ^c (n))$.
As for the complex $R\Gamma_c (G_\RR, X (\CC), (2\pi i)^n\,\RR)$, it stays for
the \term{$G_\RR$-equivariant cohomology with compact support} of the space of
complex points $X (\CC)$, where $G_\RR \dfn \Gal (\CC/\RR)$. Namely, $G_\RR$
acts by conjugation both on $X (\CC)$ and on the coefficients $(2\pi i)^n\,\RR$,
which means that the complex $R\Gamma_c (X (\CC), (2\pi i)^n\,\RR)$ carries a
natural $G_\RR$-action, and we can consider the cohomology of $G_\RR$ acting on
that complex (this is a particular case of equivariant sheaf cohomology, as
introduced by Grothendieck in the Tohoku paper). In terms of cohomology groups,
there is a spectral sequence
$$E_2^{pq} = H^p (G_\RR, H^q_c (X (\CC), (2\pi i)^n\,\RR)) \Longrightarrow H^{p+q}_c (G_\RR, X (\CC), (2\pi i)^n\,\RR).$$
In this particular case, however, we deal simply with the fixed points of the
$G_\RR$-action on $H_c^i (X (\CC), (2\pi i)^n\,\RR)$:
$$H_c^i (G_\RR, X (\CC), (2\pi i)^n\,\RR) \isom H_c^i (X (\CC), (2\pi i)^n\,\RR)^{G_\RR}$$
---this is because for any $\ZZ/2\ZZ$-module $A$, the cohomology groups
$H^p (G_\RR, A)$ are $2$-torsion for $p > 0$, and we deal with $\RR$-vector
spaces. However, for instance for integral coefficients $(2\pi i)^n\,\ZZ$,
the above spectral sequence does need to degenerate, and the groups
$H_c^i (G_\RR, X (\CC), (2\pi i)^n\,\ZZ)$ are more complicated.

% % % % % % % % % % % % % % % % % % % % % % % % % % % % % %

\section{Regulator}

Now to extract the special values of zeta functions, we also need some kind of a
regulator (generalizing the regulator of a number field). Normally in this
setting, it should be a morphism
\[ Reg\colon
  \begin{array}{c}
    \text{motivic cohomology} \\
    \text{(higher Chow groups)}
  \end{array}\text{of }X
  \longrightarrow
  \text{Deligne (co)homology of }X(\CC). \]

We use the construction of Matt Kerr, James Lewis, and Stefan Müller-Stach from
\cite{Kerr-Lewis-Muller-Stach-2006}. I won't recall the details on Deligne
(co)homology here, because it turns out that at the end of the day, thanks to
our assumption $n < 0$, things simplify tremendously, and we obtain a morphism
\[ Reg\colon R\Gamma(X_\text{\it ét},\ZZ^c (n)) \to
  R\Gamma_\text{\it BM} (G_\RR,X(\CC),(2\pi i)^n\,\RR)[1]. \]
Unfortunately, the construction of Kerr, Lewis, and Müller-Stach works under
a rather severe restriction.

\begin{drawback*}
  We need to assume that $X_\CC$ is smooth and quasi-projective.
\end{drawback*}

However, since in our particular case of $n < 0$, the regulator has a simpler
target ($G_\RR$-equivariant Borel–Moore homology of $X(\CC)$), one might wonder
if there is a simpler definition, requiring less from $X$.

\begin{question*}
  Is it possible to define a regulator in our setting under less restrictive
  assumptions on $X$?
\end{question*}

The regulator is supposed to satisfy the following condition.

\vspace{1em}

\noindent\textbf{Conjecture $\mathbf{B}(X, n)$.}
\emph{The $\RR$-dual to the regulator map
  $$Reg^\vee\colon R\Gamma_c (G_\RR, X(\CC), (2\pi i)^n\,\RR)[-1] \to
  \RHom(R\Gamma(X_\text{\it ét}, \ZZ^c (n)), \RR)$$
  is a quasi-isomorphism of complexes.}

\pagebreak

Now let's consider the morphism
\[ R\Gamma_\text{\it W,c} (X, \ZZ(n)) \otimes \RR
  \xrightarrow{\smile\theta}
  R\Gamma_\text{\it W,c} (X, \ZZ(n)) \otimes \RR [1] \]
defined by

\[ \begin{tikzcd}
    R\Gamma_\text{\it W,c} (X, \ZZ (n)) \otimes \RR \ar{d}{\isom} \\
    \RHom (R\Gamma (X_\text{\it ét}, \ZZ^c (n)), \RR) [-1] ~\oplus~ R\Gamma_c (G_\RR, X (\CC), (2\pi i)^n\,\RR) [-1] \ar[twoheadrightarrow]{d} \\
    R\Gamma_c (G_\RR, X (\CC), (2\pi i)^n\,\RR) [-1] \ar{d}{Reg^\vee}\\
    \RHom (R\Gamma (X_\text{\it ét}, \ZZ^c (n)), \RR) \ar[rightarrowtail]{d} \\
    \RHom (R\Gamma (X_\text{\it ét}, \ZZ^c (n)), \RR) ~\oplus~ R\Gamma_c (G_\RR, X (\CC), (2\pi i)^n\,\RR) \ar{d}{\isom} \\
    R\Gamma_\text{\it W,c} (X, \ZZ (n)) \otimes \RR [1]
  \end{tikzcd} \]

\begin{proposition*}
  Assume that the conjectures $\mathbf{L}^c (X_\text{\it ét},n)$ and
  $\mathbf{B} (X,n)$ hold. Then the above morphism $\smile\theta$ turns
  $H_\text{\it W,c}^\bullet (X, \ZZ(n)) \otimes \RR$ into an acyclic complex of
  finite dimensional vector spaces
  \[ \cdots \to H_\text{\it W,c}^i (X, \ZZ(n)) \otimes \RR \xrightarrow{\smile\theta}
    H_\text{\it W,c}^{i+1} (X, \ZZ(n)) \otimes \RR \xrightarrow{\smile\theta}
    H_\text{\it W,c}^{i+2} (X, \ZZ(n)) \otimes \RR \to \cdots \]
\end{proposition*}

\noindent (this is actually clear from the above definition, once we assume that
$Reg^\vee$ is a quasi-isomorphism.)

\vspace{1em}

We now use determinants of complexes, defined by Finn Knudsen and David Mumford
in \cite{Knudsen-Mumford-76}. In the generality we need, the determinant
canonically associates to a perfect complex of $R$-modules a free $R$-module of
rank $1$:
$$C^\bullet \rightsquigarrow \det\nolimits_R C^\bullet.$$
The determinant is functorial on the subcategory given by perfect complexes and
\emph{isomorphisms} in $\categ{D} (R\categ{-Mod})$, it is compatible with
distinguished triangles in a suitable sense, with base change, etc. Without
getting into details, let me just say that the properties of the determinant
imply that the long exact sequence from the last proposition induces a canonical
isomorphism
$$\lambda\colon \RR \xrightarrow{\isom} (\det\nolimits_\ZZ R\Gamma_\text{\it W,c} (X,\ZZ(n)) \otimes \RR,$$
allowing us to treat $\det\nolimits_\ZZ R\Gamma_\text{\it W,c} (X,\ZZ(n))$ as
a lattice in $\RR$.

\pagebreak

% % % % % % % % % % % % % % % % % % % % % % % % % % % % % %

\section{The main conjecture about special values}

Armed with the morphism $\lambda$, we are ready to state our conjecture about
vanishing orders and special values of the zeta function.

\vspace{1em}

\noindent\textbf{Conjecture $\mathbf{C} (X,n)$.} {\it For an arithmetic scheme $X$ and $n < 0$

\begin{enumerate}
\item[a)] assume that the conjecture $\mathbf{L}^c (X_\text{\it ét}, n)$ holds;

\item[b)] assume that $X_\CC$ is smooth, quasi-projective, so that the regulator
  morphism $Reg$ exists; assume that the conjecture $\mathbf{B}(X, n)$ holds;

\item[c)] assume that $\zeta (X, s)$ has a meromorphic continuation near
  $s = n$.
\end{enumerate}

Then

\begin{enumerate}
\item[1)] the special value $\zeta^* (X, n)$ is given up to sign by
  $$\lambda (\zeta^* (X, n)^{-1}) \cdot \ZZ = \det\nolimits_\ZZ R\Gamma_\text{\it W,c} (X, \ZZ(n)).$$

\item[2)] the vanishing order of $\zeta (X, n)$ at $s = n$ is given by the
  weighted alternating sum of ranks of the corresponding Weil-étale cohomology
  groups:
  $$\ord_{s=n} \zeta (X,s) = \sum_{i\in\ZZ} (-1)^i \cdot i\cdot \rk_\ZZ H^i_\text{\it W,c} (X, \ZZ(n)).$$
\end{enumerate}}

\vspace{1em}

Of course, one can define any kind of complexes and formulate any conjectures
about them. The conjecture $\mathbf{C} (X,n)$ is plausible because when $X$ is
regular and proper, then it is equivalent to the conjectures stated by Flach and
Morin in \cite{Flach-Morin-16}, and they showed in particular that for smooth
schemes their special value conjecture is compatible with the
\term{Tamagawa number conjecture} of Bloch, Kato, Fontaine, and Perrin-Riou
(see \cite{Fontaine-Perrin-Riou}).

Even for some easy examples, it is not trivial at all to calculate Weil-étale
cohomology and verify $\mathbf{C} (X,n)$ directly: among other things, that
would require calculation of motivic cohomology. Let us see a couple of examples
for the vanishing orders, as it is much easier to count ranks of groups.
It is easy to check that under the assumptions a) and b) made in the conjecture,
we have

\[ \sum_{i\in\ZZ} (-1)^i \cdot i\cdot \rk_\ZZ H^i_\text{\it W,c} (X, \ZZ(n)) =
  \sum_{i\in\ZZ} (-1)^i \, \dim_\RR H^i_c (G_\RR, X (\CC), (2\pi i)^n\,\RR). \]
Namely, we may use the splitting
\[
  R\Gamma_\text{\it W,c} (X,\ZZ(n)) \otimes \RR \isom
  \RHom (R\Gamma (X_\text{\it ét}, \ZZ^c (n)), \RR) [-1]
  \oplus
  R\Gamma_c (G_\RR, X (\CC), (2\pi i)^n\,\RR) [-1] \]
and the conjectural quasi-isomorphism
\[ Reg^\vee\colon R\Gamma_c (G_\RR, X(\CC), (2\pi i)^n\,\RR) [-1]
  \xrightarrow{\isom}
  \RHom(R\Gamma(X_\text{\it ét}, \ZZ^c (n)), \RR). \]
This all means that the conjecture actually says that
$$\ord_{s=n} \zeta (X,s) = \chi (R\Gamma_c (G_\RR, X(\CC), (2\pi i)^n\,\RR))$$
is the usual Euler characteristic of a very specific and computable complex.

\begin{enumerate}
\item[1)] For the case of a number ring $X = \Spec \O_F$, the space $X(\CC)$
  consists of $r_1 + 2\,r_2$ points, corresponding to the real places of $F$ and
  complex places coming in conjugate pairs:

  \[ \begin{tikzpicture}
      \matrix(m)[matrix of math nodes, row sep=1em, column sep=1em, text height=1ex, text depth=0.2ex]{
        ~ & ~ & ~ & ~ & ~ & \bullet & \bullet & \cdots & \bullet \\
        \bullet & \bullet & \cdots & \bullet \\
        ~ & ~ & ~ & ~ & ~ & \bullet & \bullet & \cdots & \bullet \\};

      \draw[->] (m-2-1) edge[loop above,min distance=10mm] (m-2-1);
      \draw[->] (m-2-2) edge[loop above,min distance=10mm] (m-2-2);
      \draw[->] (m-2-4) edge[loop above,min distance=10mm] (m-2-4);

      \draw[->] (m-1-6) edge[bend left] (m-3-6);
      \draw[->] (m-1-7) edge[bend left] (m-3-7);
      \draw[->] (m-1-9) edge[bend left] (m-3-9);

      \draw[->] (m-3-6) edge[bend left] (m-1-6);
      \draw[->] (m-3-7) edge[bend left] (m-1-7);
      \draw[->] (m-3-9) edge[bend left] (m-1-9);

      \draw ($(m-3-1)!.5!(m-3-4)$) node[yshift=-2em,anchor=base] {$r_1$ points};
      \draw ($(m-3-6)!.5!(m-3-9)$) node[yshift=-2em,anchor=base] {$2\,r_2$ points};
    \end{tikzpicture} \]

  The complex $R\Gamma_c (X(\CC), (2\pi i)^n\,\RR)$ in this case has just a
  single $\RR$-vector space in degree $0$, namely
  $$V \dfn ((2\pi i)^n\,\RR)^{\oplus r_1} \oplus ((2\pi i)^n\,\RR \oplus (2\pi i)^n\,\RR)^{\oplus r_2},$$
  where $G_\RR$ acts on $((2\pi i)^n\,\RR)^{\oplus r_1}$ by complex conjugation,
  while on $((2\pi i)^n\,\RR \oplus (2\pi i)^n\,\RR)^{\oplus r_2}$ the action is
  given by $(z_1,z_2) \mapsto (\overline{z_2}, \overline{z_1})$ on each summand
  $(2\pi i)^n\,\RR \oplus (2\pi i)^n\,\RR$. We see that
  \[ \chi (R\Gamma_c (G_\RR, X (\CC), (2\pi i)^n\,\RR)) =
    \dim_\RR V^{G_\RR} = \begin{cases}
      r_2, & n \text{ odd},\\
      r_1 + r_2, & n \text{ even},
    \end{cases} \]
  and this agrees with vanishing orders of the Dedekind zeta function
  $\zeta (\Spec \O_F, s)$ at strictly negative integers.

\item[2)] If $X$ is a variety over $\FF_q$, then
  $$\zeta (X,s) = Z (X, q^{-s}) = \exp \left(\sum_{k\ge 1} \frac{\# X (\FF_{q^k})}{k}\,q^{-ks}\right)$$
  is Weil's zeta function, which has no zeros or poles for $s < 0$. This is not
  quite obvious, but it may be seen from the Weil conjectures that if $s$ is a
  pole or zero, then it should satisfy
  $$\Re s = i/2, \quad 0 \le i \le 2\,\dim X$$
  (see e.g. \cite[p. 26--27]{Katz-Motives}). This agrees with the fact that
  trivially,
  $$\chi (R\Gamma_c (G_\RR, X (\CC), (2\pi i)^n\,\RR)) = 0.$$
\end{enumerate}

% % % % % % % % % % % % % % % % % % % % % % % % % % % % % %

\section{Stability properties}

So we stated the conjecture $\mathbf{C} (X,n)$, and it is equivalent to the
conjectures of Flach and Morin for $X$ regular, proper. This is not a big deal,
because from the very beginning, the construction of
$R\Gamma_\text{\it W,c} (X,\ZZ(n))$ follows theirs, with the difference that in
the case $n < 0$ certain things actually become simpler. Now I will explain how
new results may be obtained.

\vspace{1em}

The following properties are clear from the definition of the zeta function of
an arithmetic scheme:

\begin{enumerate}
\item[1)] If $U \hookrightarrow X \leftarrow Z$ is an open-closed decomposition,
  then
  $$\zeta (X, s) = \zeta (U, s) \cdot \zeta (Z, s).$$

\item[2)] For $r \ge 0$, for the affine space $\AA^r_X \dfn \AA_\ZZ^r \times X$
  $$\zeta (\AA^r_X, s) = \zeta (X, s - r).$$
\end{enumerate}

This suggests the following result.

\begin{theorem*}
  ~
  \begin{enumerate}
  \item[1)] If $U \hookrightarrow X \leftarrow Z$ is an open-closed
    decomposition of an arithmetic scheme, then if two out of three conjectures
    $\mathbf{C} (U, n)$, $\mathbf{C} (X, n)$, $\mathbf{C} (Z, n)$ hold, then the
    other one holds as well.

  \item[2)] The conjecture $\mathbf{C} (\AA^r_X, n)$ is equivalent to
    $\mathbf{C} (X, n-r)$.
  \end{enumerate}
\end{theorem*}

Again, it will be easier to explain this for vanishing orders. As I mentioned,
the conjecture actually says that
$$\ord_{s=n} \zeta (X,s) = \chi (R\Gamma_c (G_\RR, X(\CC), (2\pi i)^n\,\RR)).$$
In 1), we have the following distinguished triangle for $G_\RR$-equivariant
cohomology with compact support:
\[ R\Gamma_c (G_\RR, U(\CC), (2\pi i)^n\,\RR) \to
  R\Gamma_c (G_\RR, X(\CC), (2\pi i)^n\,\RR) \to
  R\Gamma_c (G_\RR, Z(\CC), (2\pi i)^n\,\RR) \to \cdots [1] \]
and therefore
\[ \chi (R\Gamma_c (G_\RR, X(\CC), (2\pi i)^n\,\RR)) =
  \chi (R\Gamma_c (G_\RR, U(\CC), (2\pi i)^n\,\RR)) +
  \chi (R\Gamma_c (G_\RR, Z(\CC), (2\pi i)^n\,\RR)). \]
In 2), we need to check that

\[ \tag{*} \chi (R\Gamma_c (G_\RR, \CC^r \times X(\CC), (2\pi i)^n\,\RR)) =
  \chi (R\Gamma_c (G_\RR, X(\CC), (2\pi i)^{n-r}\,\RR)), \]

\noindent and in fact, we have a $G_\RR$-equivariant quasi-isomorphism of
complexes
\[ R\Gamma_c (\CC^r \times X(\CC), (2\pi i)^n\,\RR) \quiso
  R\Gamma_c (X(\CC), (2\pi i)^{n-r}\,\RR) [-2r], \]
where the shift by $-2r$ does not affect the Euler characteristic.

\vspace{1em}

As for the special values part of the conjecture, in 1) one needs to use the
``Borel--Moore'' triangle
\[ R\Gamma (Z_\text{\it ét}, \ZZ^c (n)) \to
  R\Gamma (X_\text{\it ét}, \ZZ^c (n)) \to
  R\Gamma (U_\text{\it ét}, \ZZ^c (n)) \to \cdots [1] \]
which in turn gives
\[ \RHom(R\Gamma(U_\text{\it ét}, \ZZ^c (n)), \RR) \to
  \RHom(R\Gamma(X_\text{\it ét}, \ZZ^c (n)), \RR) \to
  \RHom(R\Gamma(Z_\text{\it ét}, \ZZ^c (n)), \RR) \to \cdots [1] \]
This may be combined with the above triangle for
$R\Gamma_c (G_\RR, (-)(\CC), (2\pi i)^n\,\RR)$ and the splittings
\[ R\Gamma_\text{\it W,c} (-,\ZZ(n)) \otimes \RR \isom
  \RHom (R\Gamma ((-)_\text{\it ét}, \ZZ^c (n)), \RR) [-1]
  \oplus
  R\Gamma_c (G_\RR, (-) (\CC), (2\pi i)^n\,\RR) [-1] \]
(those are not canonical, but may be chosen in a way compatible with the
triangles).

In 2), the key idea is that if $p\colon \AA^r_X \to X$ is the canonical
projection, then there is a quasi-isomorphism of complexes of sheaves on
$X_\text{\it ét}$
$$R p_* \ZZ^c (n) \quiso \ZZ^c (n - r) [2r],$$
so that
\[ R\Gamma (\AA^r_{X,\text{\it ét}}, \ZZ^c (n)) \quiso
  R\Gamma (X_\text{\it ét}, \ZZ^c (n - r)) [2r]. \]
This corresponds to the formula (*) (the shifts differ by a sign because (*) is
written for cohomology with compact support, while the above formula is morally
for the dual ``Borel--Moore homology'').

\begin{center}
  \noindent *\quad *\quad *
\end{center}

All this means that with the established machinery, we can take as a starting
point certain schemes for which the conjecture is true (e.g. schemes for which
the Tamagawa number conjecture is known), and then, using operations like
open-closed decompositions and affine bundles, construct new schemes, possibly
singular, for which the conjecture holds as well.

% % % % % % % % % % % % % % % % % % % % % % % % % % % % % %

\appendix
\section{Definition of Weil-étale complexes}

Here I outline the present construction of Weil-étale complexes for the case
$n < 0$. Following \cite{Flach-Morin-16}, we consider the complex of torsion
sheaves on $X_\text{\it ét}$
\[ \ZZ (n) \dfn \text{``}\QQ/\ZZ\text{''} (n) [-1] \dfn
  \bigoplus_p \dirlim_r  j_{p!} \mu_{p^r}^{\otimes n} [-1]. \]
Here $j_p\colon X [1/p] \to X$ is the canonical open immersion, by $\mu_{p^r}$
we denote the sheaf of roots of unity on $X [1/p]_\text{\it ét}$, and its twist
by $n < 0$ is defined by
$$\mu_{p^r}^{\otimes n} \dfn \iHom_{X[1/p]} (\mu_{p^r}^{\otimes (-n)}, \ZZ/p^r).$$

Then the definition of Weil-étale cohomology is summarized by the following
diagram in the derived category of abelian groups:

\[ \begin{tikzcd}[column sep=small, font=\small]
    & & R\Gamma_\text{\it W,c} (X, \ZZ (n))\ar{d} \\
    \RHom (R\Gamma (X_\text{\it ét}, \ZZ^c (n)), \QQ [-2]) \ar{r}{\alpha_{X,n}}\ar{d} & R\Gamma_c (X_\text{\it ét}, \ZZ (n)) \ar{r}\ar{d}{u_\infty^*} & R\Gamma_\text{\it fg} (X, \ZZ (n)) \ar{r}\ar[dashed]{d}{i_\infty^*} & \RHom (R\Gamma (X_\text{\it ét}, \ZZ^c (n)), \QQ [-1])\ar{d} \\
    0\ar{r} & R\Gamma_c (G_\RR, X (\CC), (2\pi i)^n\,\ZZ) \ar{r}{\idid} & R\Gamma_c (G_\RR, X (\CC), (2\pi i)^n\,\ZZ)\ar{d}\ar{r} & 0 [1] \\
    & & R\Gamma_\text{\it W,c} (X, \ZZ (n)) [1] \\
  \end{tikzcd} \]

Here is how this diagram is built.

\begin{enumerate}
\item[1)] Using a duality theorem from \cite{Geisser-10}, we define a morphism
  in the derived category of abelian groups
  \[ \alpha_{X,n}\colon \RHom (R\Gamma (X_\text{\it ét}, \ZZ^c (n)), \QQ [-2]) \to
    R\Gamma_c (X_\text{\it ét}, \ZZ (n)). \]

\item[2)] We pick a cone of $\alpha_{X,n}$ and call it
  $R\Gamma_\text{\it fg} (X, \ZZ (n))$.

\item[3)] Then we define a canonical morphism of complexes
  \[ u_\infty^*\colon R\Gamma_c (X_\text{\it ét}, \ZZ (n)) \to
    R\Gamma_c (G_\RR, X (\CC), (2\pi i)^n\,\ZZ) \]
  (it is a kind of comparison morphism between étale and singular cohomology),
  and check that
  $$u_\infty^*\circ \alpha_{X,n} = 0$$
  in the derived category. This implies that there is a morphism
  \emph{in the derived category}
  \[ i_\infty^*\colon R\Gamma_\text{\it fg} (X, \ZZ (n)) \to
    R\Gamma_c (G_\RR, X (\CC), (2\pi i)^n\,\ZZ) \]
  (see the diagram).

\item[4)] We pick a mapping fiber of $i_\infty^*$ and call it
  $R\Gamma_{\text W,c} (X,\ZZ (n))$.
\end{enumerate}

In step 2), it is possible to represent $R\Gamma_\text{\it fg} (X, \ZZ (n))$ by
a canonical complex. In step 3), in fact there is a unique $i_\infty^*$ sitting
in the commutative diagram, but the outlined argument gives it only
\emph{as a morphism in the derived category}, not a genuine morphism of
complexes. Therefore, $R\Gamma_{\text W,c} (X,\ZZ (n))$ is defined only up to
a non-canonical isomorphism. This is bad not only for aesthetical reasons, but
causes technical problems in practice. This is quite unsatisfactory, but works
for our purposes, because $\det\nolimits_\ZZ R\Gamma_\text{\it W,c} (X,\ZZ(n))$
is canonically defined.

\begin{question*}
  Is there a canonical way to define $R\Gamma_\text{\it W,c} (X,\ZZ(n))$?
\end{question*}

As all the problems come from the non-functoriality of cones, working instead
with stable $\infty$-categories (see \cite{Lurie-DAG-1}) might be helpful here.

\vspace{1em}

Here are some properties of the complexes and morphisms in the diagram from the
previous page:

\begin{enumerate}
\item[a)] The groups $H_c^i (X_\text{\it ét}, \ZZ (n))$ are $\QQ/\ZZ$-dual to
  finitely generated abelian groups. In particular, we have
  $R\Gamma_c (X_\text{\it ét}, \ZZ (n)) \otimes \RR \quiso 0$.

\item[b)]
  $H_\text{\it fg}^i (X, \ZZ (n)) \dfn H^i (R\Gamma_\text{\it fg} (X, \ZZ (n)))$
  are finitely generated abelian groups (hence the notation ``\emph{fg}'').

  Moreover, $H_\text{\it fg}^i (X, \ZZ (n)) = 0$ for $i \ll 0$ and it is a
  finite $2$-torsion group for $i \gg 0$. The $2$-torsion comes from the real
  points $X (\RR)$.

\item[c)] The morphism $i_\infty^*$ is torsion in the derived category;
  in particular, $i_\infty^* \otimes \RR = 0$.

\item[d)] $R\Gamma_c (X (\CC), (2\pi i)^n\,\ZZ)$ is a perfect complex. As for
  the equivariant cohomology $R\Gamma_c (G_\RR, X (\CC), (2\pi i)^n\,\ZZ)$,
  then, unless $X (\RR) = \emptyset$, the groups
  $H_c^i (G_\RR, X (\CC), (2\pi i)^n\,\ZZ)$ have finite $2$-torsion in
  arbitrarily high degrees, coming from the cohomology of
  $G_\RR \isom \ZZ/2\ZZ$. It has the same nature as the $2$-torsion in
  $H_\text{\it fg}^i (X, \ZZ (n))$, and in fact $H^i (i_\infty^*)$ is an
  isomorphism for $i \gg 0$. As a result, the complex
  $R\Gamma_\text{\it W,c} (X, \ZZ (n))$ is bounded.
\end{enumerate}

Once we tensor the diagram on the previous page with $\RR$, thanks to a) and c),
we obtain

\[ \begin{tikzcd}[column sep=small, font=\small]
    & & R\Gamma_\text{\it W,c} (X, \ZZ (n))\otimes \RR\ar{d} \\
    \RHom (R\Gamma (X_\text{\it ét}, \ZZ^c (n)), \RR [-2]) \ar{r} & 0 \ar{r} & R\Gamma_\text{\it fg} (X, \ZZ (n))\otimes \RR \ar{r}{\isom}\ar{d}{0} & \RHom (R\Gamma (X_\text{\it ét}, \ZZ^c (n)), \RR [-1]) \\
    & & R\Gamma_c (G_\RR, X (\CC), (2\pi i)^n\,\RR)\ar{d} \\
    & & R\Gamma_\text{\it W,c} (X, \ZZ (n))\otimes \RR [1] \\
  \end{tikzcd} \]

\noindent this explains the splitting
\[ R\Gamma_\text{\it W,c} (X,\ZZ(n)) \otimes \RR \isom
  \RHom (R\Gamma (X_\text{\it ét}, \ZZ^c (n)), \RR) [-1]
  \oplus
  R\Gamma_c (G_\RR, X (\CC), (2\pi i)^n\,\RR) [-1]. \]

\pagebreak

\bibliography{../these}
\bibliographystyle{../amsalpha-cust}

\end{document}
