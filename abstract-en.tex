\chapter*{Abstract}
\addcontentsline{toc}{chapter}{Abstract}

This work is dedicated to interpreting in cohomological terms the special values
of zeta functions of arithmetic schemes.

This is a part of the program envisioned and started by Stephen Lichtenbaum
(see e.g. Ann. of Math. vol. 170, 2009), and the conjectural underlying
cohomology theory is known as Weil-étale cohomology.  Later on Baptiste Morin
and Matthias Flach gave a construction of Weil-étale cohomology using Bloch's
cycle complex and stated a precise conjecture for the special values of proper
regular arithmetic schemes at any integer argument $s=n$. The goal is to extend
the above mentioned result and conjecture to special values of arbitrary
arithmetic schemes (possible singular or non-proper) while restricting to the
case $n<0$.

Following the ideas of Flach and Morin, the Weil-étale complexes are defined for
$n < 0$ for arbitrary arithmetic schemes, under standard conjectures about
finite generation of motivic cohomology. Then it is stated as a conjecture how
these complexes are related to the special values. For proper and regular
schemes, this conjecture is equivalent to the conjecture of Flach and Morin,
which also corresponds to the Tamagawa number conjecture.

We prove that the conjecture stated in this work is compatible with the
decomposition of an arbitrary scheme into an open subscheme and its closed
complement. We also show that this conjecture for an arithmetic scheme $X$ at
$s=n$ is equivalent to the conjecture for $\AA^r_X$ at $s=n-r$, for any
$r\geq 0$. It follows that, taking as an input the schemes for which the
conjecture is known, it is possible to construct new schemes, possibly singular
or non-proper, for which the conjecture holds as well. This is the main
unconditional outcome of the machinery developed in this thesis.

\ifdutch
\else
\vspace{1em}
\noindent\textbf{Keywords}: arithmetic geometry, zeta functions, Weil-étale
cohomology, special values, motivic cohomology.

This thesis was written at the Institute of Mathematics of Bordeaux and the
Institute of Mathematics of Leiden.
\fi
