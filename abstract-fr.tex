\chapter*{Résumé}
\addcontentsline{toc}{chapter}{Résumé}
\markboth{Résumé}{Résumé}

\begin{otherlanguage}{french}
  Ce travail est dédié à l'interprétation en termes cohomologiques des valeurs
  spéciales des fonctions zêta des schémas arithmétiques. C'est une partie d'un
  programme envisagé et initié par Stephen Lichtenbaum (voir par ex.
  Ann. of Math. vol. 170, 2009), et la théorie cohomologique sous-jacente
  s'appelle la cohomologie Weil-étale. Plus tard, Baptiste Morin et Matthias
  Flach ont donné une construction de la cohomologie Weil-étale en utilisant les
  complexes de cycles de Bloch, et ont énoncé une conjecture précise pour les
  valeurs spéciales des schémas arithmétiques propres et réguliers, en tout
  entier $s=n$. Le but de cette thèse est de généraliser le résultat et la
  conjecture mentionnés ci-dessus aux valeurs spéciales des schémas
  arithmétiques arbitraires (éventuellement singuliers ou non-propres) lorsque
  l'on se restreint au cas $n<0$.

  Suivant les idées de Flach et Morin, les complexes Weil-étale sont définis
  pour $n < 0$ pour les schémas arithmétiques arbitraires, sous des conjectures
  standards sur la génération finie de la cohomologie motivique. Ensuite, il est
  énoncé comme une conjecture de quelle manière ces complexes sont liés aux
  valeurs spéciales. Pour les schémas propres et réguliers, cette conjecture est
  équivalente a la conjecture de Flach et Morin, qui correspond aussi à la
  conjecture du nombre de Tamagawa.

  On prouve que la conjecture énoncée dans ce travail est compatible avec la
  décomposition d'un schéma arbitraire en un sous-schéma ouvert et son
  complémentaire fermé. On montre aussi que cette conjecture pour un schéma
  arithmétique $X$ en $s=n$ est équivalente à cette même conjecture pour
  $\AA^r_X$ en $s=n-r$, pour tout $r\geq 0$. Il suit que, en partant des schémas
  pour lesquels la conjecture est connue, on peut construire de nouveaux
  schémas, éventuellement singuliers ou non-propres, pour lesquels la conjecture
  est également vraie. C'est le principal résultat inconditionnel issu de la
  machinerie développée dans cette thèse.

  \ifdefined\dutch
  \else
  \pagebreak
  \noindent\textbf{Mots-clés}: géométrie arithmétique, fonctions zêta,
  cohomologie Weil-étale, valeurs spéciales, cohomologie motivique.

  Cette thèse a été préparée à l'Institut des Mathématiques de Bordeaux et
  l'Institut des Mathématiques de Leyde (Pays-Bas).
  \fi
\end{otherlanguage}
