\chapter*{Samenvatting}
\addcontentsline{toc}{chapter}{Samenvatting}

%% Kindly translated by Bas Edixhoven:
\begin{otherlanguage}{dutch}
  Dit werk is gewijd aan het interpreteren in cohomologische termen van de
  speciale waarden van zeta-functies van aritmetische schema's. Dit is deel van
  een programma voorgesteld en gestart door Stephen Lichtenbaum
  (zie bijvoorbeeld Ann.~of Math. vol.~170, 2009), en de conjecturale
  cohomologie-theorie staat bekend als Weil-étale cohomologie. Later gaven
  Baptiste Morin en Matthias Flach een constructie van Weil-étale cohomologie
  gebruikmakend van het cykelcomplex van Bloch, en stelden zij een precies
  vermoeden op voor de speciale waarden in willekeurige gehele getallen $s=n$
  van zeta functies van propere reguliere aritmetische schema's. Het doel van
  dit proefschrift is het bovengenoemde resultaat en vermoeden uit te breiden
  naar speciale waarden van willekeurige aritmetische schema's (mogelijk
  singulier of niet-proper) onder de beperking dat $n<0$.

  In navolging van de ideeën van Flach en Morin definiëren we Weil-étale
  complexen voor $n<0$ voor willekeurige aritmetische schema's, onder
  standaardvermoedens over eindige voortgebrachtheid van motivische
  cohomologie. Vervolgens formuleren we een vermoeden hoe deze complexen
  gerelateerd zijn een speciale waarden. Voor propere en reguliere schema's is
  dit vermoeden equivalent aan dat van Flach en Morin, dat ook correspondeert
  met het zogenaamde `Tamagawa getal vermoeden'.

  We bewijzen dat ons vermoeden compatibel is met decomposities van een
  willekeurig schema in een open deelschema en het gesloten complement ervan.
  We laten ook zien dat het vermoeden voor een aritmetisch schema $X$ in $s=n$
  equivalent is met het vermoeden voor $\AA^r_X$ in $s=n-r$, voor elke
  $r\geq0$. Daaruit volgt dat, vanuit schema's waarvoor het vermoeden bekend is,
  het mogelijk is nieuwe schema's, mogelijk singulier of niet-proper, te
  construeren waarvoor het vermoeden dan ook waar is. Dit is de het
  belangrijkste gevolg dat onafhankelijk is van vermoedens, van de in dit
  proefschrift ontwikkelde machinerie.
\end{otherlanguage}
