% % % % % % % % % % % % % % % % % % % % % % % % % % % % % %
\chapter*{Introduction}
\addcontentsline{toc}{chapter}{Introduction}
\markboth{Introduction}{Introduction}

% % % % % % % % % % % % % % % % % % % % % % % % % % % % % %

\iffalse
\epigraph{\foreignlanguage{russian}{Господи, как мир волшебен,\\
Как всё в мире хорошо.\\
Я пою богам молебен,\\
Я стираюсь в порошок\\
Перед видом столь могучих,\\
Столь таинственных вещей,\\
Что проносятся на тучах\\
В образе мешка свечей.\\
Боже мой, всё в мире пышно,\\
Благолепно и умно.\\
Богу молятся неслышно\\
Море, лось, кувшин, гумно,\\
Свечка, всадник, человек...}}{\foreignlanguage{russian}{Александр Введенский, 1930-е}}
\fi

Let $X$ be an \term{arithmetic scheme}, i.e. separated and of finite type over
$\Spec \ZZ$. The corresponding \term{zeta function} is defined by the infinite
product
$$\zeta (X,s) \dfn \prod_{x\in X_0} \frac{1}{1 - N (x)^{-s}},$$
where $X_0$ denotes the set of closed points of $X$, and $N (x)$ denotes the
cardinality of the residue field at $x\in X_0$. This infinite product converges
for $\Re s > \dim X$, and conjecturally, it has a meromorphic continuation to
the whole complex plane. I refer to \cite{Serre-65} for the basic results and
conjectures.

This thesis is concerned with studying the special values of $\zeta (X,s)$: the
goal is to interpret in cohomological terms the vanishing orders and leading
Taylor coefficients at $s = n \in \ZZ$. This is a part of the program that was
envisioned by Stephen Lichtenbaum and initiated in
\cite{Lichtenbaum-05,Lichtenbaum-09-Euler,Lichtenbaum-09-number-rings}, and the
conjectural underlying cohomology theory is known as Weil-étale
cohomology. Later on Matthias Flach and Baptiste Morin gave a construction of
Weil-étale cohomology using Bloch cycle complexes $\ZZ (n)$ to study
$\zeta (X, s)$ at $s = n \in \ZZ$, see \cite{Morin-14} and
\cite{Flach-Morin-16}. Their work concerns proper regular arithmetic schemes,
and the goal of this thesis is to relax these restrictions while studying the
case $n < 0$.

\vspace{1em}

From now on $n$ denotes a strictly negative integer.

\vspace{1em}

In chapter \ref{chapter:preliminaries} I collect various definitions and results
that are used in the constructions. Most of this material is quite
standard. This chapter is lengthy, but it is needed to set up the stage.

\vspace{1em}

Chapter \ref{chapter:Weil-etale-complexes} is dedicated to a construction of
Weil-étale complexes
$$R\Gamma_\text{\it W,c} (X,\ZZ(n)).$$
This will be done in two steps: first I construct complexes
$R\Gamma_\text{\it fg} (X, \ZZ (n))$, which by definition give a cone of certain
morphism
\[ \alpha_{X,n}\colon \RHom (R\Gamma (X_\text{\it ét}, \ZZ^c (n)), \QQ [-2]) \to
  R\Gamma_c (X_\text{\it ét}, \ZZ (n)) \]
in the derived category of complexes of abelian groups:
\begin{multline*}
  \RHom (R\Gamma (X_\text{\it ét}, \ZZ^c (n)), \QQ [-2]) \xrightarrow{\alpha_{X,n}}
  R\Gamma_c (X_\text{\it ét}, \ZZ (n)) \to
  R\Gamma_\text{\it fg} (X, \ZZ (n)) \\
  \to \RHom (R\Gamma (X_\text{\it ét}, \ZZ^c (n)), \QQ [-1])
\end{multline*}
Then I construct yet another morphism
\[ i_\infty^*\colon R\Gamma_\text{\it fg} (X, \ZZ (n)) \to
  R\Gamma_c (G_\RR, X (\CC), (2\pi i)^n\,\ZZ) \]
in the derived category and declare its mapping fiber to be
$R\Gamma_\text{\it W,c} (X,\ZZ(n))$:
\begin{multline*}
  R\Gamma_\text{\it W,c} (X,\ZZ(n)) \to
  R\Gamma_\text{\it fg} (X, \ZZ (n)) \xrightarrow{i_\infty^*}
  R\Gamma_c (G_\RR, X (\CC), (2\pi i)^n\,\ZZ) \\
  \to R\Gamma_\text{\it W,c} (X,\ZZ(n)) [1]
\end{multline*}

\vspace{1em}

Finally, in chapter \ref{chapter:regulator} I formulate the main conjecture.
I use the regulator construction from \cite{Kerr-Lewis-Muller-Stach-2006}.
After reviewing the necessary preliminaries about Deligne cohomology and
homology in \S\ref{section:deligne-cohomology}, I define in
\S\ref{section:regulator} a morphism
\[ Reg^\vee\colon R\Gamma_c (G_\RR, X (\CC), (2\pi i)^n\,\RR) [-1] \to
  \RHom (R\Gamma (X_\text{\it ét}, \ZZ^c (n)), \RR), \]
under the assumption that $X_\CC$ is smooth and quasi-projective. Then
$Reg^\vee$ is conjectured to be a quasi-isomorphism. This allows us to construct
an ad hoc ``cup product''
\[ \smile\theta\colon R\Gamma_\text{\it W,c} (X, \ZZ (n)) \otimes \RR \to
  R\Gamma_\text{\it W,c} (X, \ZZ (n)) \otimes \RR [1] \]
that gives a long exact sequence of Weil-étale cohomology groups with real
coefficients
\begin{multline*}
  \cdots \to H_\text{\it W,c}^i (X, \ZZ (n)) \otimes \RR \xrightarrow{\smile\theta}
  H_\text{\it W,c}^{i+1} (X, \ZZ (n)) \otimes \RR \\
  \xrightarrow{\smile\theta} H_\text{\it W,c}^{i+2} (X, \ZZ (n)) \otimes \RR \to \cdots
\end{multline*}
Then the general theory of determinants of complexes of Knudsen and Mumford
implies the existence of a canonical trivialization morphism
\[ \lambda\colon \RR \xrightarrow{\isom}
  (\det\nolimits_\ZZ R\Gamma_{W,c} (X,\ZZ (n))) \otimes_\ZZ \RR. \]
Our main conjecture $\mathbf{C} (X, n)$, formulated in
\S\ref{section:conjecture-C-X-n}, says that the leading Taylor coefficient of
$\zeta (X,s)$ at $s = n$ is given by
\[ \lambda (\zeta^* (X,n)^{-1})\cdot \ZZ =
  \det\nolimits_\ZZ R\Gamma_\text{\it W,c} (X, \ZZ (n)), \]
while the corresponding vanishing order is
\[ \ord_{s=n} \zeta (X,s) =
  \sum_{i\in\ZZ} (-1)^i \cdot i \cdot \rk_\ZZ H^i_\text{\it W,c} (X, \ZZ (n)). \]
If $X$ is proper and regular, then this is equivalent to Conjecture~5.12 and
Conjecture~5.13 from \cite{Flach-Morin-16}. In particular, it is showed in
\cite[\S 5.6]{Flach-Morin-16} that if $X$ is projective and smooth over a number
ring, then the special value conjecture is equivalent to the Tamagawa number
conjecture.

Finally, I verify in \S\ref{secion:stability-of-the-conjecture} that the
conjecture is compatible with the operations of taking disjoint unions of
schemes, gluing schemes from an open and closed part, and passing from $X$ to
the affine space $\AA^r_X$. This means that taking as an input the schemes for
which the conjecture $\mathbf{C} (X, n)$ is known, it is possible to construct
new schemes, possibly singular, for which the conjecture $\mathbf{C} (X, n)$
holds as well. This is the main unconditional outcome of the machinery developed
in this thesis.
