\documentclass{article}

\usepackage[utf8]{inputenc}

\usepackage{amsmath,amssymb,fullpage}

\title{Zeta-values of arithmetic schemes at negative integers and Weil-étale cohomology}
\date{}

\begin{document}

\maketitle
\thispagestyle{empty}

The aim of this project is to study special values of zeta-functions of
arbitrary arithmetic schemes (separated schemes of finite type over
$\operatorname{Spec} \mathbb{Z}$) at strictly negative integers, following the
method suggested by the conjectural Weil-étale program initiated by
Lichtenbaum and further developed by Flach--Morin. The main steps of the work
are the following.

\begin{enumerate}
\item Using Artin--Verdier duality, define Weil-étale cohomological complexes
  with compact support and coefficients in negative Tate twists
  $R\Gamma_{W,c} (X, \mathbb{Z} (n))$, at least for schemes whose étale
  hypercohomology with coefficients in Bloch's cycle complex $\mathbb{Z}^c (n)$
  is finitely generated.

\item Define the canonical map $\theta$ conceptually given by cup-product with
  the fundamental class:
  \[ R\Gamma_{W,c} (X, \mathbb{Z} (n)) \otimes \mathbb{R} \to
    R\Gamma_{W,c} (X, \mathbb{Z} (n)) \otimes \mathbb{R}[1]. \]

\item Study the functorial behavior of these Weil-étale complexes,
  i.e. covariant functoriality for open immersions and contravariant
  functoriality for proper maps. Check that the localization triangle for an
  open-closed decomposition holds and that the projective bundle formula holds.

\item Conjecture that the determinant (resp. the ``derived Euler--Poincaré
  characteristic'') of the Weil-étale complex
  $R\Gamma_{W,c} (X, \mathbb{Z}(n))$ defined by $\theta$ gives the special value
  (resp. the vanishing order) of the zeta function of the scheme $X$ at
  $s=n$. Show that this conjecture is equivalent to the Bloch--Kato conjecture
  (resp. Soulé conjecture) for smooth projective schemes over number rings
  (resp. for quasi-projective schemes over the integers). Study the
  compatibility with Geisser's conjecture over finite fields.

\item Using the stability of the conjecture by open-closed decompositions, prove
  the conjecture for a certain class $\mathcal{L} (\mathbb{Z})$ of singular
  arithmetic schemes. The simplest non-trivial case is the spectrum of a
  non-maximal order in an abelian number field at any negative integer (it seems
  like there's no cohomological description of the corresponding zeta-value in
  the literature).

\item State an equivariant refinement of the previous conjecture under the
  action of a finite group. Prove the compatibility with the Equivariant
  Tamagawa Number Conjecture of Burns--Flach. Using ``Galois descent'' try to
  prove other classes of the conjecture (stated in 4), in particular for some
  geometrically cellular schemes.
\end{enumerate}

\end{document}
