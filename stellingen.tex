\documentclass{article}

% % % % % % % % % % % % % % % % % % % % % % % % % % % % % %
% CLEAN UP THIS MESS AT SOME POINT :-(

\usepackage[leqno]{amsmath}
\usepackage{amssymb}
\usepackage{graphicx}

\usepackage{scrextend}
\usepackage{epigraph}

\renewcommand{\thefootnote}{\ifcase\value{footnote}\or*\or**\or***\or****\or($\infty$)\fi}
\usepackage{perpage}
\MakePerPage{footnote}

\usepackage{array}
\newcolumntype{x}[1]{>{\centering\hspace{0pt}}p{#1}}

\DeclareMathOperator*{\colim}{co{\lim}}
\newcommand{\dirlim}{\varinjlim}
\newcommand{\invlim}{\varprojlim}

\newcommand{\tikzpb}{\ar[phantom,pos=0.2]{dr}{\text{\large$\lrcorner$}}}
\newcommand{\tikzpbur}{\ar[phantom,pos=0.2]{dl}{\text{\large$\llcorner$}}}

\newcommand{\term}{\textbf}

\usepackage[utf8]{inputenc}

\usepackage{stmaryrd}

\usepackage{cancel}

\newcommand{\DB}{{\mathcal{D}\text{-\foreignlanguage{russian}{Б}}}}
\newcommand{\open}{\underset{\mathrm{open}}{\hookrightarrow}}
\newcommand{\closed}{\underset{\mathrm{closed}}{\hookrightarrow}}

\newcommand{\tcol}[2]{{#1 \choose #2}}

\newcommand{\homot}{\simeq}
\newcommand{\isom}{\cong}
% \newcommand{\cH}{\mathcal{H}}
\renewcommand{\hom}{\mathrm{hom}}
\renewcommand{\div}{\mathop{\mathrm{div}}}
\renewcommand{\Im}{\mathop{\mathrm{Im}}}
\renewcommand{\Re}{\mathop{\mathrm{Re}}}
\newcommand{\id}[1]{\mathrm{id}_{#1}}
\newcommand{\idid}{\mathrm{id}}

\newcommand{\ZG}{{\ZZ G}}
\newcommand{\ZH}{{\ZZ H}}

% \newcommand{\quiso}{\stackrel{\mathrm{q.iso}}{\cong}}
\newcommand{\quiso}{\simeq}

\newcommand{\personality}[1]{{\sc #1}}

\newcommand{\mono}{\rightarrowtail}
\newcommand{\epi}{\twoheadrightarrow}

\DeclareMathOperator{\Ann}{Ann}
\DeclareMathOperator{\CH}{CH}
\DeclareMathOperator{\Cl}{Cl}
\DeclareMathOperator{\Cone}{Cone}
\DeclareMathOperator{\Cop}{Cop}
\DeclareMathOperator{\Der}{Der}
\DeclareMathOperator{\Div}{Div}
\DeclareMathOperator{\D}{D}
\DeclareMathOperator{\End}{End}
\DeclareMathOperator{\Eq}{Eq}
\DeclareMathOperator{\Ext}{Ext}
\DeclareMathOperator{\Frac}{Frac}
\DeclareMathOperator{\Frob}{Frob}
\DeclareMathOperator{\Funct}{Funct}
\DeclareMathOperator{\Fun}{Fun}
\DeclareMathOperator{\GL}{GL}
\DeclareMathOperator{\Gal}{Gal}
\DeclareMathOperator{\Gr}{Gr}
\DeclareMathOperator{\Hol}{Hol}
\DeclareMathOperator{\Hom}{Hom}
\DeclareMathOperator{\Id}{Id}
\DeclareMathOperator{\Inn}{Inn}
\DeclareMathOperator{\Isom}{Isom}
\DeclareMathOperator{\Lie}{Lie}
\DeclareMathOperator{\Map}{Map}
\DeclareMathOperator{\Mat}{Mat}
\DeclareMathOperator{\Mor}{Mor}
\DeclareMathOperator{\Nat}{Nat}
\DeclareMathOperator{\Ob}{Ob}
\DeclareMathOperator{\Pic}{Pic}
\DeclareMathOperator{\Sing}{Sing}
\DeclareMathOperator{\Spec}{Spec}
\DeclareMathOperator{\Sp}{Sp}
\DeclareMathOperator{\Stab}{Stab}
\DeclareMathOperator{\Tot}{Tot}
\DeclareMathOperator{\YExt}{YExt}
\DeclareMathOperator{\ad}{ad}
\DeclareMathOperator{\card}{card}
\DeclareMathOperator{\codim}{codim}
\DeclareMathOperator{\coeq}{coeq}
\DeclareMathOperator{\coim}{coim}
\DeclareMathOperator{\coker}{coker}
\DeclareMathOperator{\cor}{cor}
\DeclareMathOperator{\eq}{eq}
\DeclareMathOperator{\ev}{ev}
\DeclareMathOperator{\fchar}{char}
\DeclareMathOperator{\gr}{gr}
\DeclareMathOperator{\im}{im}
\DeclareMathOperator{\mcd}{mcd}
\DeclareMathOperator{\ord}{ord}
\DeclareMathOperator{\pr}{pr}
\DeclareMathOperator{\rel}{rel}
\DeclareMathOperator{\res}{res}
\DeclareMathOperator{\rk}{rk}
\DeclareMathOperator{\sgn}{sgn}
\DeclareMathOperator{\supp}{supp}
\DeclareMathOperator{\trdeg}{trdeg}
\DeclareMathOperator{\tr}{tr}
\newcommand{\RHom}{R\!\Hom}
\newcommand{\RiHom}{R\underline{\Hom}}
\newcommand{\iHom}{\underline{\Hom}}

\newcommand{\legendre}[2]{\left(\frac{#1}{#2}\right)}

\newcommand{\dfn}{\mathrel{\mathop:}=}
\newcommand{\rdfn}{=\mathrel{\mathop:}}

\usepackage[rgb]{xcolor}
% \definecolor{myurlcolor}{rgb}{1.0,0.0,0.4}
% \definecolor{mylinkcolor}{rgb}{0.0,0.4,1.0}
% \definecolor{mycitecolor}{rgb}{0.0,0.4,1.0}
\definecolor{myurlcolor}{rgb}{0.0,0.0,0.0}
\definecolor{mylinkcolor}{rgb}{0.0,0.0,0.0}
\definecolor{mycitecolor}{rgb}{0.0,0.0,0.0}
\definecolor{shadecolor}{rgb}{0.89,0.88,0.80}
\definecolor{gray}{rgb}{0.7,0.7,0.7}
\definecolor{lightgray}{rgb}{0.8,0.8,0.8}

\newcommand{\refref}[2]{\hyperref[#2]{#1~\ref*{#2}}}
\newcommand{\eqnref}[1]{\hyperref[#1]{(\ref*{#1})}}

\newcommand{\tos}{\!\!\to\!\!}

\makeatletter
\newcommand{\dashfill}{%
\leavevmode \cleaders \hb@xt@ .50em{\hss -\hss }\hfill \kern \z@
% original \dotfill macro:
% \leavevmode \cleaders \hb@xt@ .44em{\hss .\hss }\hfill \kern \z@
}
\makeatother

\newcommand{\cequiv}{\simeq}

\usepackage[sc]{mathpazo}
\linespread{1.05}         % Palladio needs more leading (space between lines)
\usepackage[T2A,T1]{fontenc}

\usepackage{helvet}
\usepackage{titlesec}
\titleformat{name=\chapter}[display]
  {\normalfont\sffamily\huge\bfseries}
  {\chaptertitlename\ \thechapter}{20pt}{\Huge}
\titleformat{\section}
  {\normalfont\sffamily\Large\bfseries}
  {\thesection}{1em}{}
\titleformat{\subsection}
  {\normalfont\sffamily\large\bfseries}
  {\thesection}{1em}{}

\newcommand{\CC}{\mathbb{C}}
\newcommand{\FF}{\mathbb{F}}
\newcommand{\HH}{\mathbb{H}}
\newcommand{\NN}{\mathbb{N}}
\newcommand{\PP}{\mathbb{P}}
\newcommand{\QQ}{\mathbb{Q}}
\newcommand{\RR}{\mathbb{R}}
\newcommand{\ZZ}{\mathbb{Z}}
\renewcommand{\AA}{\mathbb{A}}

\makeatletter
\newcommand\xleftrightarrow[2][]{%
  \ext@arrow 9999{\longleftrightarrowfill@}{#1}{#2}}
\newcommand\longleftrightarrowfill@{%
  \arrowfill@\leftarrow\relbar\rightarrow}
\makeatother

% % % % % % % % % % % % % % % % % % % % % % % % % % % % % %

\usepackage{amsthm}

\newcommand{\examplesymbol}{$\blacktriangle$}
\renewcommand{\qedsymbol}{$\blacksquare$}

\newtheoremstyle{myplain}
  {\topsep}   % ABOVESPACE
  {\topsep}   % BELOWSPACE
  {\itshape}  % BODYFONT
  {0pt}       % INDENT (empty value is the same as 0pt)
  {\bfseries} % HEADFONT
  {.}         % HEADPUNCT
  {5pt plus 1pt minus 1pt} % HEADSPACE
  {\thmnumber{#2}. \thmname{#1}\thmnote{ (#3)}}   % CUSTOM-HEAD-SPEC 

\newtheoremstyle{myplainnameless}
  {\topsep}   % ABOVESPACE
  {\topsep}   % BELOWSPACE
  {\normalfont}  % BODYFONT
  {0pt}       % INDENT (empty value is the same as 0pt)
  {\bfseries} % HEADFONT
  {.}         % HEADPUNCT
  {5pt plus 1pt minus 1pt} % HEADSPACE
  {\thmnumber{#2}}   % CUSTOM-HEAD-SPEC 

\newtheoremstyle{mydefinition}
  {\topsep}   % ABOVESPACE
  {\topsep}   % BELOWSPACE
  {\normalfont}  % BODYFONT
  {0pt}       % INDENT (empty value is the same as 0pt)
  {\bfseries} % HEADFONT
  {.}         % HEADPUNCT
  {5pt plus 1pt minus 1pt} % HEADSPACE
  {\thmnumber{#2}. \thmname{#1}\thmnote{ (#3)}}   % CUSTOM-HEAD-SPEC

%%%%%%%%%%%%%%%%%%%%%%%%%%%%%%%%%%%%%%%%%%%%%%%%%%%%%%%%%%%%%%%%%%%%%%

\theoremstyle{myplain}
\newtheorem{proposition}{Proposition}[section]
\newtheorem{fact}{Fact}[section]
\newtheorem{lemma}[proposition]{Lemma}
\newtheorem{observation}[proposition]{Observation}
\newtheorem{question}[proposition]{Question}

\newtheorem{theorem}[proposition]{Theorem}
\newtheorem{conjecture}[proposition]{Conjecture}
\newtheorem{corollary}[proposition]{Corollary}

\theoremstyle{myplainnameless}
\newtheorem{nameless}[proposition]{}

\theoremstyle{mydefinition}
\newtheorem{examplex}[proposition]{Example}
\newenvironment{example}
  {\pushQED{\qed}\renewcommand{\qedsymbol}{\examplesymbol}\examplex}
  {\popQED\endexamplex}

\newtheorem*{examplexx}{Example}
\newenvironment{example*}
  {\pushQED{\qed}\renewcommand{\qedsymbol}{\examplesymbol}\examplexx}
  {\popQED\endexamplexx}

\newtheorem{definition}[proposition]{Definition}
\newtheorem{remark}[proposition]{Remark}

\makeatletter
\newcommand{\xRightarrow}[2][]{\ext@arrow 0359\Rightarrowfill@{#1}{#2}}
\makeatother

%%%%%%%%%%%%%%%%%%%%%%%%%%%%%%%%%%%%%%%%%%%%%%%%%%%%%%%%%%%%%%%%%%%%%%

\newcommand{\Et}{\mathop{\text{\rm \'Et}}}

\DeclareFontFamily{OT1}{pzc}{}
\DeclareFontShape{OT1}{pzc}{m}{it}{<-> s * [1.1] pzcmi7t}{}
\DeclareMathAlphabet{\mathpzc}{OT1}{pzc}{m}{it}

\newcommand{\separator}{\vspace{1em}\hrule\vspace{1em}}

\usepackage{pifont}
\newcommand{\categ}[1]{\text{\bf #1}}
\newcommand{\vcateg}{\mathcal}
\newcommand{\bone}{{\boldsymbol 1}}
\newcommand{\bDelta}{{\boldsymbol\Delta}}
\newcommand{\bsquare}{\textrm{\ding{114}}}
\newcommand{\bR}{{\mathbf{R}}}

% \usepackage{ulem}
\newcommand{\uuline}[1]{\underline{\underline{#1}}}

\renewcommand{\mathcal}{\mathpzc}

\makeatletter
\def\iddots{\mathinner{\mkern1mu\raise\p@
\vbox{\kern7\p@\hbox{.}}\mkern2mu
\raise4\p@\hbox{.}\mkern2mu\raise7\p@\hbox{.}\mkern1mu}}
\makeatother


\newcommand{\ssincl}{\reflectbox{\rotatebox[origin=c]{45}{$\subseteq$}}}

\newcommand{\vsupseteq}{\reflectbox{\rotatebox[origin=c]{-90}{$\supseteq$}}}
\newcommand{\vin}{\reflectbox{\rotatebox[origin=c]{90}{$\in$}}}

\newcommand{\Ga}{\mathbb{G}_\mathrm{a}}
\newcommand{\Gm}{\mathbb{G}_\mathrm{m}}

% \usepackage[a-1b]{pdfx}
\usepackage{hyperref}
\hypersetup{colorlinks=true,linkcolor=mylinkcolor,urlcolor=myurlcolor,citecolor=mycitecolor,pdfencoding=unicode}

\renewcommand{\O}{\mathcal{O}}

\usepackage[russian,spanish,francais,dutch,english]{babel}

\usepackage{tikz-cd}
\usetikzlibrary{babel}
\usetikzlibrary{arrows}
\usetikzlibrary{calc}


\usepackage{fullpage}

\begin{document}

\begin{center}
  \sf {\Large\noindent \textbf{Stellingen}

  }

  \vspace{0.5em}

  \noindent behorende bij het proefschrift

  \vspace{0.5em}

  \noindent {\Large \textbf{Zeta-values of arithmetic schemes at negative
      integers\\ and Weil-étale cohomology}

  }

  \vspace{0.5em}

  \noindent van Alexey Beshenov
\end{center}

In everything what follows, $X$ is an arithmetic scheme (separated, of finite
type over $\Spec \ZZ$) and $n$ is a \emph{strictly negative} integer.

We denote by $\ZZ^c (n)$ the dualizing Bloch's cycle complex of sheaves on
$X_\text{\it ét}$, and by $\ZZ (n)$ the complex of sheaves
$\bigoplus_p \dirlim_r j_{p!} \mu_{p^r}^{\otimes n} [-1]$, where
$j_p\colon X [1/p] \hookrightarrow X$ is the canonical open immersion for each
prime $p$ and $\mu_{p^r}^{\otimes n}$ is the sheaf of $p^r$-th roots of unity on
$X[1/p]_\text{\it ét}$ twisted by $n$.

We denote by $R\Gamma_c (X_\text{\it ét}, \mathcal{F}^\bullet)$ the étale
cohomology with compact support and by
$R\widehat{\Gamma}_c (X_\text{\it ét}, \mathcal{F}^\bullet)$ the modified étale
cohomology with compact support, as defined e.g. in Milne's book ``Arithmetic
Duality theorems''.

For brevity, we write $[A^\bullet, B^\bullet]$ instead of
$\RHom (A^\bullet, B^\bullet)$.

\vspace{1em}

All the main constructions are done assuming the \term{conjecture}
$\mathbf{L}^c (X_\text{\it ét}, n)$:
\emph{the cohomology groups $H^i (X_\text{\it ét}, \ZZ^c (n))$ are finitely
  generated for all $i \in \ZZ$}.

\begin{enumerate}
\item[I.] Assuming $\mathbf{L}^c (X_\text{\it ét}, n)$, there is a
  quasi-isomorphism of complexes
  $$R\widehat{\Gamma}_c (X_\text{\it ét}, \ZZ (n)) \xrightarrow{\isom}
[R\Gamma (X_\text{\it ét}, \ZZ^c (n)), \QQ/\ZZ [-2]].$$

\item[II.] Assume $\mathbf{L}^c (X_\text{\it ét}, n)$ and let $\alpha_{X,n}$ be
  the composition of morphisms of complexes
  \[
    [R\Gamma (X_\text{\it ét}, \ZZ^c (n)), \QQ [-2]] \to
    [R\Gamma (X_\text{\it ét}, \ZZ^c (n)), \QQ/\ZZ[-2]] \xleftarrow{\isom}
    R\widehat{\Gamma}_c (X_\text{\it ét}, \ZZ (n)) \to
    R\Gamma_c (X_\text{\it ét}, \ZZ (n))
  \]
  Let $R\Gamma_\text{\it fg} (X, \ZZ (n))$ be a cone of $\alpha_{X,n}$:
  \[
    [R\Gamma (X_\text{\it ét}, \ZZ^c (n)), \QQ [-2]] \xrightarrow{\alpha_{X,n}}
    R\Gamma_c (X_\text{\it ét}, \ZZ (n)) \to
    R\Gamma_\text{\it fg} (X, \ZZ (n)) \to
    [R\Gamma (X_\text{\it ét}, \ZZ^c (n)), \QQ [-1]]
  \]
  Then the cohomology groups $H^i (R\Gamma_\text{\it fg} (X, \ZZ (n)))$ are
  finitely generated, trivial for $i \ll 0$, and only have $2$-torsion for
  $i \gg 0$.

\item[III.] For any prime $\ell$ the group
  $H^i_c (X_{\overline{\QQ},\text{\it ét}}, \QQ_\ell/\ZZ_\ell (n))^{G_\QQ}$
  has no nontrivial divisible elements.

\item[IV.] Assume $\mathbf{L}^c (X_\text{\it ét}, n)$ and let $\alpha_{X,n}$ be
  as above. Let
  \[ u_\infty^*\colon R\Gamma_c (X_\text{\it ét}, \ZZ (n)) \to
    R\Gamma_c (G_\RR, X (\CC), (2\pi i)^n\,\ZZ) \]
  be the canonical comparison morphism, discussed in \S\S 0.7--0.8 of the
  thesis. Then $u_\infty^*\circ \alpha_{X,n} = 0$. Let $i_\infty^*$ be a morphism
  of complexes defined via
  \[ \begin{tikzcd}
      {[R\Gamma (X, \ZZ^c (n)), \QQ [-2]]} \ar{r}{\alpha_{X,n}}\ar{d} & R\Gamma_c (X_\text{\it ét}, \ZZ (n)) \ar{d}{u_\infty^*}\ar{r} & R\Gamma_\text{\it fg} (X, \ZZ (n)) \ar[dashed]{d}{i_\infty^*}\ar{r} & \cdots\ar{d} \\
      0\ar{r} & R\Gamma_c (G_\RR, X (\CC), (2\pi i)^n\,\ZZ) \ar{r}{\idid} & R\Gamma_c (G_\RR, X (\CC), (2\pi i)^n\,\ZZ) \ar{r} & 0
    \end{tikzcd} \]
  and let $R\Gamma_\text{\it W,c} (X,\ZZ(n))$ be a mapping fiber of
  $i_\infty^*$:
  \[
    R\Gamma_\text{\it W,c} (X,\ZZ(n)) \to
    R\Gamma_\text{\it fg} (X, \ZZ (n)) \xrightarrow{i_\infty^*}
    R\Gamma_c (G_\RR, X (\CC), (2\pi i)^n\,\ZZ) \to
    R\Gamma_\text{\it W,c} (X,\ZZ(n)) [1]
  \]
  Then $R\Gamma_\text{\it W,c} (X,\ZZ(n))$ is a perfect complex.
\end{enumerate}

\begin{center}
  \noindent * ~ * ~ * ~ * ~ *
\end{center}

To formulate the next result, we denote by $\mathbf{C} (X,n)$ the following
conjecture.

{\it
  \begin{enumerate}
  \item[a)] assume that the conjecture $\mathbf{L}^c (X_\text{\it ét}, n)$ holds;

  \item[b)] assume that $X_\CC$ is smooth, quasi-projective, so that the regulator
    morphism
    \[ Reg\colon R\Gamma (X_\text{\it ét}, \ZZ^c (n)) \to
      R\Gamma_{BM} (G_\RR, X (\CC), (2\pi i)^n\,\RR) [1] \]
    exists and assume the \term{regulator conjecture}: the $\RR$-dual is a
    quasi-isomorphism
    \[ Reg^\vee\colon R\Gamma_c (G_\RR, X (\CC), (2\pi i)^n\,\RR) [-1] \xrightarrow{\isom}
      [R\Gamma (X_\text{\it ét}, \ZZ^c (n)), \RR]. \]

  \item[c)] assume that the zeta-function $\zeta (X,s)$ has a meromorphic
    continuation near $s=n$.
  \end{enumerate}

  \textbf{Then}

  \begin{enumerate}
  \item[1)] the leading coefficient $\zeta^* (X,n)$ of the Taylor expansion of
    $\zeta (X,s)$ at $s = n$ is given up to sign by
    \[ \lambda (\zeta^* (X,n)^{-1})\cdot \ZZ =
      \det\nolimits_\ZZ R\Gamma_\text{\it W,c} (X, \ZZ (n)), \]
    where $\lambda$ is the trivialization morphism defined using the regulator in
    \S 2.3 of the thesis;

  \item[2)] the vanishing order of $\zeta (X,n)$ at $s = n$ is given by the
    weighted alternating sum of ranks of $H^i_\text{\it W,c} (X, \ZZ (n))$:
    \[ \ord_{s=n} \zeta (X,s) =
      \sum_{i\in\ZZ} (-1)^i \cdot i \cdot \rk_\ZZ H^i_\text{\it W,c} (X, \ZZ (n)). \]
  \end{enumerate}}

\begin{center}
  \noindent * ~ * ~ * ~ * ~ *
\end{center}

\begin{enumerate}
\item[V.] The conjecture $\mathbf{C} (X,n)$ is compatible with disjoint unions,
  open-closed decompositions and taking affine bundles in the following sense.

  \begin{itemize}
  \item If $X = \coprod_{0 \le i \le r} X_i$ is a disjoint union of arithmetic
    schemes, then the conjectures $\mathbf{C} (X_i, n)$ for $i = 0,\ldots,r$
    together imply $\mathbf{C} (X, n)$.

  \item If $U \hookrightarrow X \hookleftarrow Z$ is an open-closed decomposition
    of an arithmetic scheme, then if two out of three conjectures
    $\mathbf{C} (U, n)$, $\mathbf{C} (Z, n)$, $\mathbf{C} (X, n)$ hold, the
    other one holds as well.

  \item The conjecture $\mathbf{C} (\AA^r_X,n)$ is equivalent to
    $\mathbf{C} (X,n-r)$.
  \end{itemize}

\item[VI.] Sometimes it is possible to talk about unique cones in the derived
  category. For a distinguished triangle
  $A^\bullet \xrightarrow{u} B^\bullet \xrightarrow{v} C^\bullet \xrightarrow{w} A^\bullet[1]$
  assume that $A^\bullet$ is a complex such that $H^i (A^\bullet)$ are finite
  dimensional $\QQ$-vector spaces and $C^\bullet$ is ``almost perfect'', meaning
  that $H^i (C^\bullet)$ are finitely generated groups, zero for $i \ll 0$ and
  have only $2$-torsion for $i \gg 0$. Then the cone of $u$ is unique up to a
  unique isomorphism in the derived category.

\item[VII.] If $A$ and $B$ are finitely generated abelian groups, then up to
  equivalence, every extension of $\Hom (B,\QQ/\ZZ)$ by $\Hom (A,\QQ/\ZZ)$ is
  $\QQ/\ZZ$-dual to an extension of $A$ by $B$.

\item[VIII.] The order of zero of the Dedekind zeta function of a number field
  $K$ at $n < 0$ may be interpreted via the equivariant cohomology of
  $X = \Spec \O_K$ as
  $\ord_{s = n} \zeta_K (s) = \dim_\RR H^0_c (G_\RR, X (\CC), (2\pi i)^n\,\RR)$.
\end{enumerate}

\end{document}
